\documentclass{coursepaper}

\usepackage[utf8]{inputenc}
\usepackage[T1]{fontenc}
\usepackage[portuguese]{babel}
\usepackage{amsmath}
\usepackage{listings,lstautogobble}
\usepackage{bbm}

\author{Eduardo Parducci, Lucas Koiti}
\title{Projeto Computacional 2 - Diferenças Finitas (Relatório de saída)}
\author{Eduardo Parducci 170272\\Lucas Koiti 182579}
\studentnumber{ }
\instructor{Prof. Aurelio Oliveira}
\coursename{Cálculo numérico}
\coursenumber{MS211}
\coursesection{ D}
\college{UNICAMP}
\date{20 de Junho de 2017}

\begin{document}
    \maketitle

    \section{Tabela de saídas}
        \paragraph{}
        Foram selecionados 5 pontos em comum gerados a partir dos 3 diferentes valores de $h=0.1$, $h=0.01$ e $h=0.001$, como mostra a tabela abaixo:\\

        \begin{table}[h]
        \centering
        \caption{}
        \begin{tabular}{r|r|r|r}

        $x_{i}$ & $y_{i}, h=0.1$ & $y_{i}, h=0.01$ & $y_{i}, h=0.001$ \\
        \hline                               % para uma linha horizontal
        0.1 &   0.874066  &  0.8740914  & 8.7280e-01\\
        0.2 &   0.742678  &  0.7427455  & 7.4275e-01\\
        0.3 &   0.610499  &  0.6106165  & 6.1062e-01\\
        0.4 &   0.482123  &  0.4822899  & 4.8229e-01\\
        0.5 &   0.361898  &  0.3621041  & 3.6211e-01

        \end{tabular}
        \end{table}

        \paragraph{}
        Como podemos observar, a variação entre os resultados obtidos quando trocamos o valor de $h$ é de aproximadamente $1^-3$, podemos afirmar que o erro cometido é menor que $h^2$ em todos os casos.

        \paragraph{}
        O algoritmo demonstrou-se eficiente, uma vez que,mesmo com precisão de $h=0.001$ o tempo de processamento foi meramente perceptível, levando aproximadamente 0,5 segundo para retornar a solução desejada.




\end{document}
