\documentclass{article}

\usepackage[utf8]{inputenc}
\usepackage[T1]{fontenc}
\usepackage[portuguese]{babel}
\usepackage{amsmath}
\usepackage{listings,lstautogobble}
\usepackage{bbm}

\title{
    MS211 - Turma D\\
    \large Projeto Computacional 1 - Métodos Iterativos\\
    \large (Descrição de funcionamento)\\
    \large Prof. Aurelio Oliveira
}
\author{Eduardo Parducci - 170272\\Lucas Koiti      - 182579\\}
\date{\today}

\begin{document}

    \maketitle
    \newpage
    \newpage

    \section{Implementação das funções}
        \subsection{Gauss-Jacobi (\textit{jacobi.m})}
            A função \textit{jacobi} tem como principal objetivo encontrar o conjunto solução $X_{1 \times n}$ de um
            sistema linear utilizando o método de Jacobi-Richardson estimando uma solução inicial nula
            $X^0 = (0_{0} , 0_{1}, 0_{2},..., 0_{n})^t$.\\

            \textbf{Entradas:}
                \begin{itemize}
                  \item Matriz de coeficientes $A_{n \times n}$
                  \item Matriz solução $B_{1 \times n}$
                \end{itemize}
            \textbf{Saídas:}
                \begin{itemize}
                  \item Matriz de incógnitas $X_{1 \times n}$
                  \item Contador de iterações $k$
                \end{itemize}

            O código realiza os seguintes procedimentos:\\
            > inicia um vetor solução nulo 'x'\\
            > determina o erro como 0.0001\\
            > determina o vetor auxiliar para salvar o resultado da iteração anterior\\
            > determina uma variavel 'continua' como controle de laço, para verificar se convergiu\\
            > determina uma variavel contadora para o numero de iterações 'k'\\

            >cria um laço enquanto 'continua' for 'true'\\
              >percorre as colunas e linhas\\
                >se a linha(i) for diferente da coluna(j) ele realiza:\\
                  >>subtrai do auxiliar(i) o valor de A(i,j) multiplicado pelo x(j). [Ax = B]\\
                >atualiza o auxiliar para o valor de auxiliar(i) mais B(i) dividido pelo A(i,i)\\
              >termina a iteração\\

              >verifica se o resultado ja convergiu\\
                >>se tiver convergido, altera a variavel 'continua' para 'false'om\\
                >>caso contrario, ela continua 'true'\\

              >atualiza o vetor solução x para o mesmo valor do auxiliar, que são os resultados encontrados na ultima iteração\\

              >soma uma iteração na variavel 'k' contadora\\

              >repete o laço caso 'continua' for 'true' e encerra caso for 'false'\\

            >o resultado pode ser visualizado na tela retirando as barras de comentario %{ %} da ultima parte\\


                \subsection{Gauss-Seidel (\textit{seidel.m})}
                    A função \textit{seidel} tem como principal objetivo encontrar o conjunto solução $X_{1 \times n}$ de um
                    sistema linear utilizando o método de Seidel estimando uma solução inicial nula
                    $X^0 = (0_{0} , 0_{1}, 0_{2},..., 0_{n})^t$.\\

                    \textbf{Entradas:}
                        \begin{itemize}
                          \item Matriz de coeficientes $A_{n \times n}$
                          \item Matriz solução $B_{1 \times n}$
                        \end{itemize}
                    \textbf{Saídas:}
                        \begin{itemize}
                          \item Matriz de incógnitas $X_{1 \times n}$
                          \item Contador de iterações $k$
                        \end{itemize}

                    O código realiza os seguintes procedimentos:
                    > inicia um vetor solução nulo
                    > determina um contador 'k' para determina a quantidade de iterações
                    > determina um erro 'e' de 0.0001
                    > determina uma variavel 'continua' para condicionar a parada do laço

                    >cria um laço enquanto continua for 'true'
                      >cria uma variavel 'ant' para saber o estado anterior da solução 'x', dentro da mesma iteração
                      >percorre as linhas e colunas da matriz 'A'. [Ax = B]
                      >mantem a solução 'x' na posição (i), ou atual, como zero
                      >se a linha 'i' for diferente da coluna 'j' ele realiza:
                        >>subtrai do auxiliar(i) o valor de A(i,j) multiplicado pelo x(j). [Ax = B]
                      >após a operação, ja atualiza o vetor solução 'x' para a soma de 'x(i)' mais 'B(i)' dividido por A(i,i)
                      >termina a iteração

                      >verifica se o resultado ja convergiu
                        >>se tiver convergido, altera a variavel 'continua' para 'false'
                        >>caso contrario, ela continua 'true'

                      >soma um na variavel de iteração 'k'

                      >repete o laço caso 'continua' for 'true' e encerra caso for 'false'

                    >o resultado pode ser visualizado na tela retirando as barras de comentario %{ %} da ultima parte

\end{document}
