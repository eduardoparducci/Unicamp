\documentclass{article}

\usepackage[utf8]{inputenc}
\usepackage[T1]{fontenc}
\usepackage[portuguese]{babel}
\usepackage{amsmath}
\usepackage{listings,lstautogobble}
\usepackage{bbm}

\title{
    MS211 - Turma D\\
    \large Projeto Computacional 1 - Métodos Iterativos\\
    \large (Relatório)\\
    \large Prof. Aurelio Oliveira
}
\author{Eduardo Parducci - 170272\\Lucas Koiti      - 182579\\}
\date{\today}

\begin{document}

    \maketitle
    \newpage

    \section{Relatório}
      Imaginou-se um problema que consistia na linha de produção de uma fábrica de
      grande porte, a qual utiliza milhares de componentes para construir um produto.
      A matriz 1000x1000, portanto, seria composta pelo preço desses componentes, a fim
      de otimizar o lucro obtido pela empresa. Com essa contextualização, utilizou-se
      um programa para conseguir os dados iniciais, o qual funciona
      utilizando valores pseudo-randomicos para a solução do sistema e seus valores,
      pois Ax = B, sendo x e B valores determinados, pode-se gerar a matriz A, que
      respeita as condições de convergência. Como o objetivo era testar o desempenho
      das tecnicas de jacobi e seidel, os valores utilizados não eram "reais".
        Ao rodar as duas tecnicas, verificou que o jacobi despende de um tempo inviavel
      para matrizes de ordem muito grande, o fato de atualizar as variaveis após
      o final da iteração, transforma o programa muito lento em grandes proporções.
      O seidel, no entanto, como ele atualiza as variaveis de acordo com o que ele
      encontra no meio da iteração, transforma a execução muito rápida e, assim, mais
      eficiente que o jacobi. Em comparação, seidel precisou de 6 iterações para
      alcançar o resultado, enquanto jacobi alcançou 500 e ainda não estava perto
      de convergir para a solução desejada. Conclui-se, portanto, que o seidel é mais
      eficiente computacionalmente para matrizes de grandes ordens.
\end{document}
